\documentclass[a4paper]{article}
\usepackage{styles}
\usepackage{amssymb, amsthm}
\usepackage{amsmath}
\usepackage[landscape]{geometry}
\usepackage{multicol}
\usepackage{mathtools}
% \usepackage{quattrocento}
\usepackage[shortlabels]{enumitem}
\usepackage{extarrows}
\usepackage{dsfont} % Indicator
\usepackage{listings}
\usepackage{mdframed}

\newtheorem{definition}{DFN}
\newtheoremstyle{mytheoremstyle} % name
    {\topsep}                    % Space above
    {\topsep}                    % Space below
    {}                   % Body font
    {}                           % Indent amount
    {\bfseries\scshape}                   % Theorem head font
    {:}                          % Punctuation after theorem head
    {.5em}                       % Space after theorem head
    {}  % Theorem head spec (can be left empty, meaning ‘normal’)
\theoremstyle{mytheoremstyle}
\newtheorem{theorem}{Teorema}
\newtheorem{lemma}{Lema}
\newtheorem{axiom}{Axioma}
\newtheorem{corollary}{Corolario}[theorem]
\newtheorem*{obs}{Obs}

% Probando

\definecolor{bcGray}{rgb}{0.95,0.95,0.92}
\newcounter{nalg}[section]
\refstepcounter{nalg}
\newenvironment{code}[1]{
    \mdfsetup{
        frametitle = Algoritmo~\thenalg: #1,
        frametitlealignment = \centering,
        frametitlerule=true,
        backgroundcolor=bcGray,
        linecolor=bcGray
    }
    \ttfamily
    \begin{mdframed}\relax
}{
    \end{mdframed}
    \refstepcounter{nalg}
}


\advance\topmargin-.8in
\advance\textheight3in
\advance\textwidth3in
\advance\oddsidemargin-1.5in
\advance\evensidemargin-1.5in
\parindent0pt
\parskip2pt
\newcommand{\hr}{\centerline{\rule{3.5in}{1pt}}}

\title{}
\author{}

% Macros útiles
\newcommand{\R}{\mathbb{R}}
\newcommand{\C}{\mathbb{C}}
\providecommand{\set}[1]{\left\{#1\right\}}
\providecommand{\abs}[1]{\left|#1\right|}
\providecommand{\norm}[1]{\left\|#1\right\|}
\providecommand{\dotp}[1]{\left\langle#1\right\rangle}
\DeclareMathOperator{\spec}{spec}

\begin{document}
\formTitle{Cálculo Numérico}
\begin{multicols*}{3}

\begin{roundbox}{Eigenvalores y Eigenvectores}
	Hola.
    \begin{theorem}
        Si $A$ es una matriz triangular en bloques, entonces $\spec(A) = \bigcup_{i=1}^{m} \spec(A_{ii})$
        \[
          \begin{bmatrix}
              A_{11} & \dots & \dots & a_{1n} \\
              0 & A_{22} & \ddots & \vdots \\
              0 & 0 & \dots & A_{nn}
          \end{bmatrix}
        \]
    \end{theorem}

    \begin{definition}[Eigenpar dominante]
        Sea $A \in \C^{n\times n}$ con eigenvalores $\lambda_1,\dots , \lambda_n$.
        El eigenvalor dominante es el valor $\lambda_i$ que satisface $\abs{\lambda_i} \geq \abs{\lambda_j}, \, \forall j \neq i$.
        El eigenvector asociado a $\lambda_i$ junto con el eigenvalor dominante forman el eigenpar dominante $(\lambda_i, v_i)$.
    \end{definition}

    \begin{theorem}
        Sea $A$ diagonalizable (hipótesis por simplicidad) y sea $(\lambda_1, v_1)$ su eigenpar dominante, y $q^0 = c_1 v_1 + \dots + c_n v_n$ \underline{con $c_1 \neq 0$}.
        Entonces, la sucesión $q^i = \frac{1}{\lambda_1} A q^{i-1}$ con $i=1,2,\dots$ converge a $c_1 v_1$.
    \end{theorem}

    \begin{code}{Power iteration}
        \label{alg:power-iter}
        {\footnotesize 1} \, $x^0 = \frac{q^0}{\norm{q^0}}$ \\
        {\footnotesize 2} \, \textbf{for} k = 1:max \\
        {\footnotesize 3} \qquad $v^k = A x^{k-1}$ \\
        {\footnotesize 4} \qquad $x^k = \frac{v^k}{\norm{v^k}}$ \\
        {\footnotesize 5} \, end
    \end{code}

    \begin{definition}[Método de potencia inversa]
        Es una variante del método de potencia, y aproxima el eigenpar más chico de $A$.
        Se aplica el algoritmo \ref{alg:power-iter} a la matriz $A^{-1}$.
    \end{definition}

    \begin{definition}[Método de potencia inversa con shift]
        El método de potencia inversa con shift introduce un shift $s$ para que el método aproxime el eigenvalor más cercano a $s$.
        Es el el método de potencia (Alg. \ref{alg:power-iter}) aplicado a la matrix $(A- sI)^{-1}$.
    \end{definition}

    \begin{definition}[Orden de convergencia]
        Se dice que un método numérico converge con potencia $\alpha$ si
        \[
            \lim_{i\to\infty} \frac{\norm{\text{\emph{error}}(i+1)}}{\norm{\text{\emph{error}}(i)}^{\alpha}} = c
        \]
    \end{definition}
\end{roundbox}

\begin{roundbox}{Eigenvalores y Eigenvectores}
    Los métodos de potencia inversa y sus variantes convergen todos linealmente con tasa $\frac{|\lambda_1|}{|\lambda_2|}$ donde $|\lambda_1| > |\lambda_2| > \dots$.
    Es decir, $\lambda_1$ es el eigenvalor más grande y $\lambda_2$ el segundo más grande.

    El shift óptimo para el método de potencia inversa con shift se calcula con el \emph{shift de Rayleigh}

    \begin{definition}[Cociente de Rayleigh]
        Definimos el cociente de Rayleigh para $q^0$ como
        \[
            \rho = \frac{\dotp{q^0, Aq^0}}{\norm{q_o}_{2}^{2}}
        \]
    \end{definition}

    El método de potencia inverso con shift dinámico de Rayleigh converge cuadráticamente.
\end{roundbox}

\end{multicols*}
\end{document}
